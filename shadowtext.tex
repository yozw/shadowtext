%% This is file `shadowtext.tex',
%% Copyright 2012 Yori Zwols
%
% This work may be distributed and/or modified under the
% conditions of the LaTeX Project Public License, either version 1.3
% of this license or (at your option) any later version.
% The latest version of this license is in
%   http://www.latex-project.org/lppl.txt
% and version 1.3 or later is part of all distributions of LaTeX
% version 2005/12/01 or later.
%
% This work has the LPPL maintenance status `maintained'.
% 
% The Current Maintainer of this work is Y. Zwols.
%
%

\documentclass{article}

\usepackage{xcolor}
\usepackage{outlinetext}
\usepackage{multicol}

\begin{document}
\title{\bf The shadowtext package \\ \bigskip \normalfont\small Version 0.3}
\author{Y. Zwols (yz2198@columbia.edu)}
\maketitle

\shadowoffset{1pt}
\parindent=0pt
\parskip=8pt

\section{Usage}
The shadowtext package introduces a new command, namely the {\tt $\backslash$shadowtext} command. This is a box
that adds a shadow behind text. Its usage is simple. The following code and its output illustrates the usage. 

\begin{minipage}{\textwidth}
\begin{multicols*}{2} 
{\footnotesize
\begin{verbatim}
\shadowtext{This is a shadowtext}
\end{verbatim}}

\shadowtext{This is a shadowtext}
\end{multicols*}
\end{minipage}

\section{Changing the position of the shadow}
The position of the shadow consists of two components: a horizontal offset and a vertical offset. 
The package offers two ways of changing the position of the shadow. The first and easiest way of 
changing the position is by using the {\tt $\backslash$setshadowoffset} command. This commands
sets the horizontal and vertical offset simultaneously. For example:

\begin{minipage}{\textwidth}
\begin{multicols*}{2} 
{\footnotesize
\begin{verbatim}
\shadowoffset{0.5pt}
\shadowtext{This is a shadowtext}
\end{verbatim}}

\shadowoffset{0.5pt}
\shadowtext{This is a shadowtext}
\end{multicols*}
\end{minipage}

Alternatively, one may use the commands {\tt shadowoffsetx} and {\tt shadowoffsety} to independently change
the horizontal and vertical offset. For example:

\begin{minipage}{\textwidth}
\begin{multicols*}{2} 
{\footnotesize
\begin{verbatim}
\shadowoffsetx{1pt}
\shadowoffsety{4pt}
\shadowtext{This is a shadow text}
\end{verbatim}}

\shadowoffsetx{1pt}
\shadowoffsety{4pt}
\shadowtext{This is a shadow text}\\
\phantom{}
\end{multicols*}
\end{minipage}

\section{Changing the color of the shadow}
The color of the shadow may be changed by using the {\tt shadowcolor} and {\tt shadowrgb} commands.\footnote{Internally, the 
package uses the {\tt $\backslash$color} command from the color package to set its color} 
The {\tt shadowcolor} command takes as an argument the name of a color, whereas {\tt shadowrgb} takes a comma-separated list of red, green, blue
values. For example, the following code adds a blueish shadow using {\tt $\backslash$shadowcolor}:

\begin{minipage}{\textwidth}
\begin{multicols*}{2} 
{\footnotesize
\begin{verbatim}
\shadowoffset{2pt}
\shadowcolor{blue!40!white}
\shadowtext{This is a shadow text}
\end{verbatim}}

\shadowoffset{2pt}
\shadowcolor{blue!40!white}
\shadowtext{This is a shadow text}\\
\phantom{}
\end{multicols*}
\end{minipage}

The following code produces a red shadow using {\tt $\backslash$shadowrgb}:

\begin{minipage}{\textwidth}
\begin{multicols*}{2} 
{\footnotesize
\begin{verbatim}
\shadowoffset{2pt}
\shadowrgb{1.0, 0.5, 0.5}
\shadowtext{This is a shadow text}
\end{verbatim}}

\shadowoffset{2pt}
\shadowrgb{1.0, 0.5, 0.5}
\shadowtext{This is a shadow text}\\
\phantom{}
\end{multicols*}
\end{minipage}

\section{More examples}
The following code illustrates how to use shadowtext in more interesting settings:

\definecolor{navy}{rgb}{0,0,0.5}

\begin{minipage}{\textwidth}
\begin{multicols*}{2} 
{\footnotesize
\begin{verbatim}
\definecolor{navy}{rgb}{0,0,0.5}
\shadowrgb{0.8, 0.8, 1}
\shadowoffset{1pt}
\shadowtext{
   \color{navy}
   \fontsize{16}{16}\selectfont
   Large!}
\end{verbatim}}

\color{navy}
\shadowrgb{0.8, 0.8, 1}
\shadowoffset{2pt}
\shadowtext{\fontsize{16}{16}\selectfont \textbf{Large!}}\\
\phantom{}\\
\phantom{}\\
\phantom{}
\end{multicols*}
\end{minipage}


\color{black}




\begin{minipage}{\textwidth}
\begin{multicols*}{2} 
{\footnotesize
\begin{verbatim}
\shadowoffset{1pt}
\color{black}
\shadowtext{%
   \begin{tabular}{|l|l|}
   \hline
   1 & 2 \\
   3 & 4 \\
   \hline
   \end{tabular}
}
\end{verbatim}}

\shadowoffset{1pt}
\shadowtext{
   \begin{tabular}{|l|l|}
   \hline
   1 & 2 \\
   3 & 4 \\
   \hline
   \end{tabular}
}
\phantom{}\\
\phantom{}\\
\phantom{}\\
\phantom{}\\
\phantom{}\\
\phantom{}
\end{multicols*}
\end{minipage}





\begin{minipage}{\textwidth}
\begin{multicols*}{2} 
{\footnotesize
\begin{verbatim}
\shadowoffset{1pt}
\shadowtext{$\sqrt{2\pi}e^{-n}$}
\end{verbatim}}

\shadowoffset{1pt}
\shadowtext{$\sqrt{2\pi}e^{-n}$}\\
\phantom{}
\end{multicols*}
\end{minipage}

\color{black}

\begin{minipage}{\textwidth}
\begin{multicols*}{2} 
{\footnotesize
\begin{verbatim}
\shadowoffset{2pt}
\shadowtext{
   \color{navy}%
   \fontencoding{T1}%
   \fontfamily{pag}%
   \fontseries{b}%
   \fontsize{32}{32}%
   \selectfont%
   Section \color{red}\arabic{section}%
}
\end{verbatim}}

\shadowoffset{2pt}
\shadowtext{
   \color{navy}%
   \fontencoding{T1}%
   \fontfamily{pag}%
   \fontseries{b}%
   \fontsize{32}{32}%
   \selectfont%
   Section \color{red}\arabic{section}
}
\phantom{}\\
\phantom{}\\
\phantom{}\\
\phantom{}\\
\end{multicols*}
\end{minipage}


\end{document}
